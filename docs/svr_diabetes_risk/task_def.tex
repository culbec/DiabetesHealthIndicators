\documentclass[a4paper]{article}
\pagenumbering{gobble}

\usepackage{style}

% Keywords command
\providecommand{\keywords}[1]
{
  \small	
  \textbf{\textit{Keywords---}} #1
}

\title{
	Prediction of Diabetes Risk according to Social \& Health Indicators \\
	\large \textit{Supervised Regression}
}

\author{Alexandru Profir$^{1}$ \\
        \small $^{1}$Babe\cb{s}-Bolyai University
}

%% \date{} % Comment this line to show today's date

\begin{document}

\maketitle
\hspace{10pt}

\section{Problem definition}

\par The purpose of this particular approach is to develop a framework capable of predicting the \textit{diabetes risk score} based on a set of individual health indicators. This predictive capability can assist in preventive healthcare and early detection, helping individuals and practitioners make informed medical decisions.

\section{Problem specification}

\par This section presents the specification of the problem, by analyzing the input and output features, as well as pre and post conditions applied to them.

\subsection{Input Data}

\par The input data consists of $100,000$ patient profiles with features based on demographics, lifestyle habits, family history, and clinical measurements that are well-established indicators of diabetes risk. The data is generated using statistical distributions from real-world medical research, ensuring \textit{privacy preservation} while reflecting realistic health patterns.

\par Table~\ref{table-input-data} presents a description of all input features used in this problem. This table showcases the name of the feature, the type of the feature, a brief description, and the domain (\textit{value range}) of the presented feature. As confirmed by the creators of this \href{https://www.kaggle.com/datasets/mohankrishnathalla/diabetes-health-indicators-dataset}{dataset}, all values fall within realistic medical ranges, matching realistic population health patterns, being a viable option when exploring lifestyle and clinical health patterns.

\begin{table}[H]
	\centering
	\small
	\begin{tabular}{|l|l|p{5cm}|p{4cm}|}
		\hline
		\textbf{Feature} & \textbf{Type} & \textbf{Description} & \textbf{Domain} \\
		\hline
		patient\_id & Integer & Unique patient identifier & $\mathbb{N} \cap [1, 100000]$ \\
		\hline
		age & Integer & Age of patient in years & $\mathbb{N} \cap [18, 90]$ \\
		\hline
		gender & Categorical & Patient gender & \{Male, Female, Other\} \\
		\hline
		ethnicity & Categorical & Ethnic background & \{White, Hispanic, Black, Asian, Other\} \\
		\hline
		education\_level & Categorical & Highest completed education & \{No formal, Highschool, Graduate, Postgraduate\} \\
		\hline
		income\_level & Categorical & Income category & \{Low, Medium, High\} \\
		\hline
		employment\_status & Categorical & Employment type & \{Employed, Unemployed, Retired, Student\} \\
		\hline
		smoking\_status & Categorical & Smoking behavior & \{Never, Former, Current\} \\
		\hline
		alcohol\_consumption\_per\_week & Float & Drinks consumed per week & $\mathbb{R_+} \cap [0, 30]$ \\
		\hline
		physical\_activity\_minutes\_per\_week & Integer & Physical activity (weekly minutes) & $\mathbb{N} \cap [1, 100000]$ \\
		\hline
		diet\_score & Integer & Diet quality (higher = healthier) & $\mathbb{N} \cap [1, 10]$ \\
		\hline
		sleep\_hours\_per\_day & Float & Average daily sleep hours & $\mathbb{R_+} \cap [3, 12]$ \\
		\hline
		screen\_time\_hours\_per\_day & Float & Average daily screen time hours & $\mathbb{R_+} \cap [0, 12]$ \\
		\hline
		family\_history\_diabetes & Integer & Family history of diabetes & $\mathbb{N} \cap \{0, 1\}$ \\
		\hline
		hypertension\_history & Integer & Hypertension history & $\mathbb{N} \cap \{0, 1\}$ \\
		\hline
		cardiovascular\_history & Integer & Cardiovascular history & $\mathbb{N} \cap \{0, 1\}$ \\
		\hline
		bmi & Float & Body Mass Index ($kg/m^2$) & $\mathbb{R_+} \cap [15, 45]$ \\
		\hline
		waist\_to\_hip\_ratio & Float & Waist-to-hip ratio & $\mathbb{R_+} \cap [0.7, 1.2]$ \\
		\hline
		systolic\_bp & Integer & Systolic blood pressure ($mmHg$) & $\mathbb{N} \cap [90, 180]$ \\
		\hline
		diastolic\_bp & Integer & Diastolic blood pressure ($mmHg$) & $\mathbb{N} \cap [60, 120]$ \\
		\hline
		heart\_rate & Integer & Resting heart rate ($bpm$) & $\mathbb{N} \cap [50, 120]$ \\
		\hline
		cholesterol\_total & Float & Total cholesterol ($mg/dL$) & $\mathbb{R_+} \cap [120, 30]$ \\
		\hline
		hdl\_cholesterol & Float & HDL cholesterol ($mg/dL$) & $\mathbb{R_+} \cap [20, 100]$ \\
		\hline
		ldl\_cholesterol & Float & LDL cholesterol ($mg/dL$) & $\mathbb{R_+} \cap [50, 200]$ \\
		\hline
		triglycerides & Float & Triglycerides ($mg/dL$) & $\mathbb{R_+} \cap [50, 500]$ \\
		\hline
		glucose\_fasting & Float & Fasting glucose ($mg/dL$) & $\mathbb{R_+} \cap [70, 250]$ \\
		\hline
		glucose\_postprandial & Float & Post-meal glucose ($mg/dL$) & $\mathbb{R_+} \cap [90, 350]$ \\
		\hline
		insulin\_level & Float & Blood insulin level ($\mu U/mL$) & $\mathbb{R_+} \cap [2, 50]$ \\
		\hline
		hba1c & Float & HbA1c (\%) & $\mathbb{R_+} \cap [4, 14]$ \\
		\hline
	\end{tabular}
	
	\caption{Input data description}
	\label{table-input-data}
\end{table}

\subsection{Output Data}

\par This problem proposes to solve the issue of predicting the diabetes risk score, based on social \& health indicators. The \textit{diabetes risk score} is a computed integer value based on the proposed social \& health indicators, ranging between $[0, 100]$. Based on the proposed features, the solution will provide a \textit{diabetes risk score} within the same range, indicating if a patient has a risk to diabetes or not.

\section{Learning Task}

\par This section presents the formalization of the learning task regarding the presented problem, accounting the features proposed to solve the problem and the goal provided by the solution. The specification of the learning task is presented below:

\begin{itemize}
	\item \textbf{Task}: prediction of the diabetes risk score;
	\item \textbf{Performance}: quality of the resulted prediction, represented by how accurate is the prediction with respect to the real value;
	\item \textbf{Experience}: prediction of the diabetes risk score on a large set of labeled instances;
\end{itemize}

\par The above described task will be solved through an empirical analysis, driven by data patterns rather than explicit programming.

%TC:ignore
% \keywords{federative learning, machine learning, privacy, security, trust, ethics}
%TC:endignore

%\nocite{*}
%\bibliographystyle{alpha}
%\bibliography{references}

\end{document}